\documentclass{article}
\usepackage{amsmath}
\usepackage{amsthm}
\usepackage{amssymb}
\usepackage{algorithm}
\usepackage[noend]{algpseudocode}
\usepackage{polski}
\usepackage[utf8]{inputenc}

\newtheorem{theorem}{Twierdzenie}
\newtheorem{observation}{Obserwacja}

\title{Faktoryzacja Lyndona\\ \large Na podstawie ''Factorizing Words over an Ordered Alphabet '' -- Duval}
\author{Krzysztof Pióro}
\date{Maj 2022}

\begin{document}

\maketitle

\section{Wstęp}

Słowo będziemy nazywać \textit{prostym} (lub słowem Lyndona) jeśli jest ściśle mniejsze od wszystkich swoich nietrywialnych przesunięć cyklicznych.

\textbf{Faktoryzacją Lyndona} słowa $w$ nazwiemy podział słowa $w = w_1w_2\ldots w_k$ taki, że wszystkie słowa $w_i$ są proste oraz zachodzi
$w_1 \geq w_2 \geq \cdots \geq w_k$. Taka faktoryzacja zawsze istnieje i jest unikalna.

Algorytm Duvala konstruuje faktoryzację Lyndona w czasie liniowym i stałej dodatkowej pamięci.

\section{Własności faktoryzacji Lyndona}

\begin{observation}
    Słowo $w$ jest \textit{proste} wtedy i tylko wtedy, gdy jest ściśle mniejsze niż wszystkie swoje nietrywialne sufiksy 
\end{observation}

\begin{proof}
    Załóżmy, że słowo $w$ jest ściśle mniejsze niż wszystkie swoje nietrywialne sufiksy oraz, że nie jest \textit{proste}.
    Wtedy mamy takie dwa słowa $u, v$, że $w = uv$, $vu \leq uv$ ($vu$ jest świadkiem dla faktu, że $w$ nie jest \textit{proste}) oraz $uv < v$ 
    ($uv$ jest ściśle mniejsze niż wszystkie inne sufiksy). Ale z $uv < v$ możemy wywnioskować $uv < vu$, czyli mamy sprzeczność.

    W drugą stronę załóżmy, że słowo $w$ jest \textit{proste} oraz, że nie jest ściśle mniejsze od wszystkich swoich nietrywialnych sufiksów.
    Wtedy mamy takie słowa $u, v$, że $w = uv$, $uv < vu$ ($w$ jest słowem \textit{prostym}) oraz $v \leq uv$. Rozważmy dwa przypadki:
    \begin{itemize}
        \item $v$ nie jest prefiksem $uv$ -- wtedy $vu < uv$, czyli sprzeczność
        \item $v$ jest prefiksem $uv$ -- tutaj zauważamy, że $u$ jest najmniejsze leksykograficznie spośród prefiksów długości $|u|$ przesunięć cyklicznych słowa $w$.
        Zatem $vu \leq uv$, czyli sprzeczność.
    \end{itemize} 
\end{proof}

\begin{observation}
    Słowa \textit{proste} nie posiadają właściwych prefikso-sufiksów.
\end{observation}

\begin{observation}
    Dla faktoryzacji Lyndona $w = w_1w_2\ldots w_k$ słowo $w_k$ jest minimalnym sufiksem słowa $w$. 
\end{observation}

\begin{theorem}
    Dla każdego słowa $w$ istnieje faktoryzacja Lyndona.
\end{theorem}

\begin{proof}
    Pojedyncza litera jest słowem \textit{prostym}, zatem możemy zacząć od podziału słowa $w$ na pojedyncze literki.
    Łatwo zauważyć, że dla dwóch słów prostych $u$, $v$ takich, że $u < v$ słowo $uv$ również jest słowem prostym.
    Zatem dopóki będą istniały dwa sąsiednie słowa $u$, $v$ w naszej faktoryzacji takie, że $u < v$ to możemy je łączyć w jedno słowo $uv$.
    Powtarzając tą procedurę otrzymamy faktoryzację Lyndona.
\end{proof}

\begin{theorem}
    Dla każdego słowa $w$ istnieje \textbf{unikalna} faktoryzacja Lyndona.
\end{theorem}

\begin{proof}
    Z poprzedniego twierdzenia wiemy, że faktoryzacja Lyndona zawsze istnieje. Pozostało nam pokazać jej unikalność.
    Wykorzystamy do tego celu fakt, że ostatnie słowo z faktoryzacji Lyndona słowa $w$ jest minimalnym sufiksem słowa $w$. 
    Możemy zatem odciąć minimalny sufiks słowa $w$ i wywołać się indukcyjnie na krótszym słowie.
\end{proof}

\section{Algorytm Duvala}

Zacznijmy od wprowadzenia dodatkowej definicji. Słowo nazwiemy \textit{prawie prostym} jeśli jest postaci $w = u^t\bar{u}$, gdzie 
$\bar{u}$ jest prefiksem słowa $u$ (być może pustym). 

Dodatkowo udowodnimy teraz, że słowo \textit{prawie proste}
ma tylko jeden okres $u$, który jest słowem \textit{prostym}. W przeciwnym wypadku mielibyśmy dwa \textit{proste} okresy $u_1, u_2$.
Załóżmy, że $|u_1| < |u_2|$. Wtedy $u_1$ jest okresem $u_2$. Ale z tego wynikałoby, że $u_2$ ma właściwy prefikso-sufiks, 
co jest sprzeczne z tym, że $u_2$ jest słowem \textit{prostym}. 

Algorytm Duvala będzie utrzymywał podział słowa wejściowego $w$ na $3$ słowa $w=v_1v_2v_3$ takie, że dla fragmentu $v_1$ faktoryzacja Lyndona jest już znana, 
a fragment $v_2$ jest słowem \textit{prawie prostym}. 

Algorytm będzie utrzymywał ten podział za pomocą dwóch zmiennych $i, j$. Zmienna $i$ będzie wskazywała na początek słowa $v_2$, a zmienna $j$ będzie wskazywała na początek słowa $v_3$. 
W każdym kroku algorytmu będziemy próbowali doczepić literę $w[j]$ do słowa $v_2$. W tym celu będziemy porównywali ją z literą słowa $v_2$ wyznaczoną przez zmienną $k$ (taką, że $j - k$ 
to długość słowa $u$ występującego w \textit{prawie prostym} słowie $v_2 = u^t\bar{u}$). 
Dokładniej będziemy mieli trzy przypadki:
\begin{itemize}
    \item $w[j] = w[k]$: dodanie $w[j]$ do $v_2$ nie narusza założenia, że $v_2$ jest \textit{prawie proste}. 
    
    W tym przypadku zwiększamy zmienne $j$ oraz $k$.
    \item $w[j] > w[k]$: słowo $v_2 + w[j]$ staje się proste. 
    Aby to udowodnić zauważmy, że minimalny sufiks słowa $u^t\bar{u}w[j]$ musi zacząć się w pierwszym słowie $u$. 
    W przeciwnym przypadku moglibyśmy rozszerczyć go o długość $|u|$ i z faktu, że $w[j] > w[k]$ otrzymalibyśmy, że ten
    dłuższy sufiks jest mniejszy. Wiemy zatem, że minimalny sufiks zaczyna się w pierwszym fragmencie $u$, 
    natomiast z faktu, że $u$ jest proste otrzymujemy, że minimalny sufiks musi zacząć się od pierwszej litery $u$, 
    czyli słowo $v_2 + w[j]$ staje się proste.

    W tym przypadku zwiększamy $j$ i ustawiamy $k$ na początek słowa $v_2$.
    \item $w[j] < w[k]$: słowo $v_2 + w[j]$ przestaje być \textit{prawie proste}.
    Aby to udowodnić załóżmy, że słowo $v_2 + w[j]$ pozostaje \textit{prawie proste}. 
    Wtedy $v_2 + w[j] = s^t\bar{s}$ dla jakiegoś słowa prostego $s$. Oczywiście $s \neq u$.
    Ponadto $s$ jest okresem słowa $v_2$, co daje nam sprzeczność z tym, że \textit{prawie proste}
    słowo ma tylko jeden \textit{prosty} okres.
    
    W tym przypadku dla naszego słowa $v_2 = u^t\bar{u}$ dzielimy $u^t$ na $t$ słów \textit{prostych},
    dodajemy je do $v_1$ (czyli wynikowej faktoryzacji Lyndona), ustawiamy zmienne $i$ oraz $k$ na początek pozostałej części słowa $v_2$ ($\bar{u}$), a zmienną $j$ na jedną pozycję dalej.
\end{itemize}





\begin{algorithm}[H]
    \caption{\textbf{Duval$(w)$}}
    \begin{algorithmic}
        \State $i := 1$
        \State factorization := empty list
        \While {$i \leq n$}
            \State $j := i+1;$  $k := i$;
            \While {$j \leq n$ \textbf{and} $w[k] \leq w[j]$}
                \If{$w[k] < w[j]$}
                    \State $k := i$
                \Else
                    \State $k := k + 1$;
                \EndIf
                \State $j := j + 1$;
            \EndWhile
           \While {$i \leq k$}
                \State factorization append $w[i\ldots i+j-k-1]$
                \State $i := i + j - k$;
           \EndWhile
        \EndWhile
        \State \Return factorization
    \end{algorithmic}
\end{algorithm}

\begin{proof}[Dowód poprawności]
    Z powyższych przypadków możemy od razu wywnioskować, że algorytm Duvala rozkłada słowo $w$ na 
    $w = w_1w_2\ldots w_k$ takie, że wszystkie $w_i$ są słowami \textit{prostymi}.
    Pozostało pokazać, że $w_1 \geq w_2 \geq \cdots \geq w_k$. 
    W tym celu zastanówmy się co dzieje się w trakcie kroku z trzeciego przypadku. Niech $a := w[j]$. 
    Zauważmy, że $\bar{u}av < u$ dla 
    dowolnego słowa $v$. Zatem wszystkie prefiksy słowa $v_2v_3$, które otrzymamy po kroku z trzeciego przypadku,
    będą ściśle mniejsze od słowa $u$, czyli od ostatniego słowa z faktoryzacji słowa $v_1$. 
    Z tego wynika, że słowa, które później będą dodane do faktoryzacji zachowają warunek monotoniczności.
\end{proof}

\begin{proof}[Złożoność algorytmu]    
    Pierwsza wewnętrzna pętla w każdym kroku zwiększa zmienną $j$. Zmienna ta
    może być jednak zmniejszana w wyniku trzeciego przypadku. Zauważmy jednak że, 
    to o ile cofniemy tą zmienną wynosi co najwyżej tyle, jak długie jest poprzednio dodane słowo Lyndona. 
    Dodatkowo za każdym razem kiedy cofamy zmienną $j$, ostatnie słowo w faktoryzacji jest inne, zatem
    możemy oszacować sumę cofnięć przez $O(n)$. Otrzymujemy więc, że ta pętla działa w czasie $O(n)$.
    
    Druga wewnętrzna pętla w algorytmie działa sumarycznie w czasie $O(n)$, bo wypisuje faktoryzację.

    Ponadto zewnętrzna pętla algorytmu nie przekroczy $n$ iteracji, 
    bo jej każda iteracja zwiększa zmienną $i$, co pokazuje, że cały algorytm działa w czasie $O(n)$.
\end{proof}

\end{document}
