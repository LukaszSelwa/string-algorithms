\documentclass[a4paper,12pt]{article}
\usepackage{amsmath}
\usepackage{amsthm}
\usepackage{amssymb}
\usepackage{algorithm}
\usepackage{algorithmic}
\usepackage{polski}
\usepackage[utf8]{inputenc}

\theoremstyle{definition}
\newtheorem{definition}{Definition}[section]
\newtheorem{theorem}{Theorem}[section]
\newtheorem{lemma}{Lemma}[section]

\DeclareMathOperator{\SA}{SA}
\DeclareMathOperator{\LCP}{LCP}
\DeclareMathOperator{\PLCP}{PLCP}
\DeclareMathOperator{\rev}{rev}

\title{Permuted Longest-Common-Prefix Array}
\author{Demian Banakh}
\date{}

\begin{document}

\maketitle

\section{Notacja}

\begin{itemize}
\item $a * b$ znaczykatenację słów $a$ i $b$.
\item Tekst wejściowy oznaczam $t[1..n]$
\item Sufiks $i$ znaczy $t[i..n]$
\item Tablica sufiksowa dla $t$ to $\SA[1..n]$
\[
t[\SA[1]..n] < t[\SA[2]..n] < \dots < t[\SA[n]..n]
\]
\item Funkcja $lcp(x, y)$ zwraca najkrótszy wspólny prefiks słów $x$ i $y$. Dla wygody będę oznaczał
\[
lcp(i, j) := lcp(t[i..n], t[j..n])
\]
\item Tablica $\LCP[1..n]$ zdefiniowana przez $t$ i $\SA$ to
\[
\LCP[j] = lcp(\SA[j - 1], \SA[j])
\]
\item Tablica $\PLCP[1..n]$ jest permutacją tablicy $\LCP$ zdefiniowaną wzorem
\begin{equation*}
\PLCP[\SA[i]] = \LCP[i]
\end{equation*}
\end{itemize}

\section{Algorytmy i dowody}

Kluczowa właśność $\PLCP$:
\begin{lemma}
$\PLCP[i] \ge \PLCP[i - 1] - 1$ dla każdego $i > 1$.
\begin{proof}
Niech $j, j'$ takie że $\SA[j] = i$ oraz $\SA[j'] = i - 1$. Mamy z definicji
\begin{align*}
\PLCP[i] = \LCP[j] &= lcp(\SA[j - 1], i) \\
\PLCP[i - 1] = \LCP[j'] &= lcp(\SA[j' - 1], i - 1)
\end{align*}
Rozważmy następujące sufiksy
\begin{align*}
A = t[\SA[j - 1] - 1..n] &= t[\SA[j - 1] - 1] * t[\SA[j - 1]..n] \\
B = t[\SA[j' - 1]..n] &= t[\SA[j' - 1]] * t[\SA[j' - 1] + 1..n] \\
C = t[i - 1..n] &= t[i - 1] * t[i..n] \\
B &\le C \text{ (poprzedni leksykograficznie)}
\end{align*}
Oczywiście $lcp(\SA[j' - 1]+1, i) \ge lcp(\SA[j' - 1], i - 1) - 1$, przy czym równość jest wtw gdy $t[\SA[j' - 1]] = t[i - 1]$. Skoro nie ma sufiksów pomiędzy $\SA[j - 1]$ a $\SA[j]=i$ w porządku leksykograficznym, to
\begin{gather*}
t[\SA[j' - 1] + 1..n] \le t[\SA[j - 1]..n] < t[i..n]
\end{gather*}
Zatem
\[
lcp(\SA[j - 1], i) \ge lcp(\SA[j' - 1] + 1, i) \ge lcp(\SA[j' - 1], i - 1) - 1
\]
\end{proof}
\end{lemma}

Ten lemat prawie wprost prowadzi do wydajnego Algorytmu A obliczenia PLCP: mając obliczoną wartość $\PLCP[i - 1]$, odejmujemy 1 i porównujemy sufiksy znak po znaku dopóki są równe. Oczywiście musimy wiedzieć który to jest sufiks $\SA[j - 1]$; do tego celu liczymy dodatkową tablicę $\Phi[\SA[j]] = \SA[j - 1]$.

\begin{algorithm}[H]
\caption{Algorytm A}
\begin{algorithmic} 
\REQUIRE $\SA$ and text
\ENSURE $\PLCP$
\FOR{$i=1$ \TO $n$}
\STATE $\Phi[\SA[i]] = \SA[i - 1]$
\ENDFOR
\STATE $l \gets 0$
\FOR{$i=1$ \TO $n$}
\STATE $s \gets \Phi[i]$
\WHILE{$text[i+l] = text[s+l]$}
\STATE $l \gets l + 1$
\ENDWHILE
\STATE $\PLCP[i] \gets l$
\STATE $l \gets \max(l - 1, 0)$
\ENDFOR
\end{algorithmic}
\end{algorithm}

Łatwo widać, że złożoność czasowa Algortymu A jest $O(n)$, bo zmniejszamy $l$ o 1 co najwyżej $n$ razy, a maksymalna wartość $l$ jest co najwyżej $n$. Z punktu widzenia pamięci, potrzebuje dodatkowej tablicy $\Phi$ rozmiaru $n$.
\par
W dość naturalny sposób można ulepszyć kontrolę nad space-time trade-off modyfikując algorytm tak, żeby produkował tylko co $q$-ty element $\PLCP$ (wystarczy obliczyć tylko co $q$-ty element $\Phi$; naiwne liczenie $\PLCP[i]$ zaczynamy od $(\PLCP[i - 1] - q)$ - wynika z Lematu 1), a obliczenie dowolnego elementu $\PLCP$ na bazie co $q$-tego było w czasie amortyzowanym $O(q)$ (dla dowolnego $i$ szukamy najbliższych takich $k$, $k+q$ na których wartość $\PLCP$ jest znana; wtedy obliczamy naiwnie $\PLCP[i]$ w interwale zadanym przez $\PLCP[k]$ i $\PLCP[k + q]$, którego długość jest średnio $O(q)$).

\begin{definition}
Nazywamy wartość $\PLCP[i] = lcp(i, \Phi[i])$ redukowalną, gdy $t[i - 1] = t[\Phi[i] - 1]$.
\end{definition}

\begin{lemma}
Jeśli $PLCP[i]$ jest redukowalna, to $PLCP[i] = PLCP[i - 1] - 1$.
\begin{proof}
Pokażemy to, kontynuując dowód Lematu 1 z dodatkowym założeniem
\[
t[i - 1] = t[\SA[j - 1] - 1]
\]
W takim razie mamy $A < C$, ponieważ $\SA[j - 1]$ i $\SA[j] = i$ to są kolejne sufiksy. Skoro $B$ i $C$ to też kolejne sufiksy
\[
A \le B < C
\]
Skoro pierwsze litery $A$ i $C$ są równe, pierwsza litera $B$ musi być taka sama. Oznaczmy $A', B', C'$ odpowiednio $A, B, C$ bez pierwszych liter. Mamy
\[
A' \le B' < C'
\]
Ale $A'$ i $C'$ to kolejne sufiksy, więc $A' = B'$, oraz $\SA[j - 1] = \SA[j' - 1] + 1$. W końcu
\[
lcp(\SA[j - 1], i) = lcp(\SA[j' - 1] + 1, i) = lcp(\SA[j' - 1], i - 1) - 1
\]
ponieważ $t[i - 1] = t[\SA[j - 1] - 1] = t[\SA[j' - 1]]$.
\end{proof}
\end{lemma}

Wnioskujemy z tego lematu, że nieredukowalne wartości $\PLCP[i]$ jednoznacznie wyznaczają pozostałe, zatem algorytm B obliczenia $\PLCP$ najpierw wylicza wszystkie nieredukowalne wartości, potem dopełnia pozostałe.

\begin{algorithm}[H]
\caption{Algorytm B}
\begin{algorithmic} 
\REQUIRE $\SA$ and text
\ENSURE $\PLCP$
\FOR{$i=1$ \TO $n - 1$}
\STATE $j \gets \SA[i]$
\STATE $k \gets \SA[i + 1]$
\IF{$text[j - 1] = text[k - 1]$}
\STATE $\PLCP[k] \gets lcp(j, k)$
\ENDIF
\ENDFOR
\FOR{$i=1$ \TO $n - 1$}
\IF{$\PLCP[i + 1] < \PLCP[i] - 1$}
\STATE $\PLCP[i + 1] \gets \PLCP[i] - 1$
\ENDIF
\ENDFOR
\end{algorithmic}
\end{algorithm}

\begin{lemma}
Suma wszystkich nieredukowalnych wartości lcp jest $\le 2n\log n$.
\begin{proof}
Niech $l = \PLCP[i] = lcp(i, j)$, gdzie $j = \Phi[i]$, jest nieredukowalną wartością. Zatem
\begin{align*}
t[i - 1] &\neq t[j - 1] \\
t[i..i+l-1] &= t[j..j+l-1] \\
t[i + l] &\neq t[j + l]
\end{align*}
Dla każdego $0 \le k \le l-1$, będziemy przydzielać koszt 1 do pary $t[i + k] = t[j + k]$ w następujący sposób.
\par
Rozpatrzmy drzewo sufiksowe słowa $\overline{t}$; niech $v_{i+k}$ i $v_{j+k}$ będą liśćmi odpowiadającymi prefiksom $t[1..i+k]$ i $t[1..j+k]$. Najniższy wspólny przodek $u$ węzłów $v_{i+k}$ i $v_{j+k}$ odpowiada $t[i..i+k]$ - te prefiksy zgadzają się na dokładnie $(k+1)$ znaków od końca. Jeśli $v_{i+k}$ jest w mniejszym poddrzewie $u$, to koszt pary $t[i + k] = t[j + k]$ przydzielamy do węzła $v_{i+k}$, wpp do $v_{j+k}$. Gdy $v_{i+k}$ został wybrany, nazwiemy $u$ \textit{drogim przodkiem} węzła $v_{i+k}$, a $v_{j+k}$ \textit{drogim bratem} węzła $v_{i+k}$ w odniesieniu do $u$, wpp analogicznie.
\par
Teraz wystarczy pokazać, że każdy liść ma koszt co najwyżej $2\log n$. W szczególności, pokażmy, że (a) każdy liść ma co najwyżej $\log n$ drogich przodków, i (b) w odniesieniu do każdego z tych drogich przodków dany liść ma co najwyżej 2 drogich braci.
\begin{itemize}
\item Rozważmy ścieżkę od $v$ do korzenia. Z konstrukcji, przy każdym drogim przodku węzła $v$ na ścieżce, poddrzewo rośnie przynajmniej w 2 razy. Zatem może być nie więcej niż $\log n$ takich przodków.
\item Niech $u$ to drogi przodek węzła $v$ i $w$ to drogi brat węzła $v$ w odniesieniu do $u$. Mamy, że $v,u,w$ odpowiadają
\[
t[1..i+k], t[i..i+k],t[1..j+k] \text{ dla pewnych $i, j$ tż  $i = \Phi[j]$ lub $j = \Phi[i]$}
\]
Bez ograniczenia ogólności, niech $i = \Phi[j]$. Załóżmy, że istnieje inny drogi brat $w' \neq w$ węzła $v$ w odniesieniu do $u$. Teraz $w'$ musi odpowiadać $t[1..j'+k]$ dla $j' = \Phi[i]$ (ponieważ $i = \Phi[j] \neq \Phi[j']$). Od razu widać, że trzeciego takiego brata nie może istnieć.
\end{itemize}
Suma kosztów wszystkich liści w drzewie (równa sumie wszystkich nieredukowalnych wartości lcp) jest ograniczona przez $2n\log n$.
\end{proof}
\end{lemma}

Z tego lematu wnioskujemy, że złożoność czasowa Algorytmu B jest $O(n\log n)$, natomiast z punktu widzenia pamięci - używa $O(1)$ poza samą tablicą $\PLCP$.
\par
Podobnie jak wcześniej, ten algorytm da się w naturalny sposób zmodyfikować tak, żeby produkował co $q$-ty element $\PLCP$, a obliczenie dowolnego na bazie co $q$-tego zajmowało średnio $O(q)$.
\par
Warto dodać, że przedstawione algorytmy obliczenia $\PLCP$ są znacznie szybsze w praktyce, niż standardowe algorytmy obliczenia $\LCP$. Zaletą Algorytmu A jest bardzo skuteczne wykorzystanie właśności \textit{locality of reference}, drugiego - wymagania pamięciowe (jeśli przedstawić $\PLCP$ jako bit-tablicę, potencjalnie można zmieścić się 3n bitów pamięci).

\end{document}