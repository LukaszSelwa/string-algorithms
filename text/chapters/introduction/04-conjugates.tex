\subsection{Słowa sprzężone}

\begin{definition}{}{}
  Słowa $v, w$ są \textbf{\textit{sprzężone}} wtedy i tylko wtedy, gdy istnieją słowa $u_1$, $u_2$ takie, że $v = u_1 u_2$ i $w = u_2 u_1$.
\end{definition}

\begin{corollary}{}{}
  Relacja sprzężenia (cyklicznego obrotu) jest relacją równoważności, więc definiuje klasy równoważności w $\A^*$.
\end{corollary}

\begin{problem}{lothaire2002algebraic}{}
  Pokaż, ile elementów ma klasa równoważności dla słowa $v$.
\end{problem}

\begin{problem}{crochemore2002jewels}{s. 17}
  Pokaż dowód małego twierdzenia Fermata na bazie wiedzy, że słowo pierwotne $v$ należy do klasy równoważności o mocy $|v|$.
\end{problem}

\begin{proof}
Przypomnijmy na wstępie dowodu wypowiedź małego twierdzenia Fermata: jeżeli $p$ jest liczbą pierwszą, a $n$ dowolną liczbą naturalną, to $p | (n^p-n)$.

Zdefiniujmy taką relację na słowach, że $x$ jest w relacji z $y$, jeżeli tylko $x$ jest cyklicznym przesunięciem $y$. Oczywiście jest to relacja równoważności. Rozważmy słowa unarne długości $p$, czyli słowa postaci $a^p$, gdzie $a$ jest pewną literą. Niech $K$ będzie zbiorem wszystkich nieunarnych słów długości $p$ nad alfabetem $\{1,\cdots,n\}$. Wszystkie te słowa są pierwotne, ponieważ ich długość jest liczbą pierwszą oraz nie są unarne. Na podstawie założeń dostajemy, że każda klasa równoważności naszej relacji ma dokładnie $p$ elementów. Dodatkowo zauważmy, że zbiór $K$ ma dokładnie $n^p-n$ elementów. Ponieważ $K$ może być podzielony na rozłączne podzbiory mające po $p$ elementów, to $p|(n^p-n)$, co kończy dowód.
\end{proof}

\begin{definition}{}{}
  Słowo $w$ jest \textbf{\textit{słowem Lyndona}} wtedy i tylko wtedy, gdy jest minimalnym (leksykograficznie, wojskowo) słowem w ramach klasy sprzężenia.
\end{definition}
