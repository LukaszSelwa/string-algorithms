\documentclass[12pt]{article}
\usepackage[utf8]{inputenc}
\usepackage{polski}
\usepackage[margin=1in]{geometry}
\usepackage{amsmath}
\usepackage{enumitem}
\usepackage{amsthm}
\usepackage{amsfonts}
\usepackage[colorlinks]{hyperref}
\usepackage{multicol}
\usepackage{algorithmicx,algpseudocode}
\usepackage{comment}

\setlength{\parindent}{0pt}

\title{Omówienie pracy\\
``A Linear Space Algorithm for the LCS Problem``\\
autorstwa S. Kiran Kumar oraz C. Pandu Rangan
}
\author{Bartłomiej Jachowicz}
\date{}

\theoremstyle{definition}
\newtheorem{definition}{Definicja}[section]

\theoremstyle{definition}
\newtheorem{theorem}{Twierdzenie}[section]

\theoremstyle{definition}
\newtheorem{lemma}{Lemat}[section]


\begin{document}

\maketitle

\section{Główna idea algorytmu}
Algorytm zaprezentowany w pracy opiera się na zastosowaniu techniki \textit{dziel i zwyciężaj}. Pozwala ona znaleźć szukany najdłuższy wspólny podciąg w złożoności $O(n(m - p))$ oraz przy wykorzystaniu liniowego rozmiaru pamięci. W poniższym opisie $A$ oraz $B$ oznaczają słowa o długości odpowiednio $m$ oraz $n$, które są argumentami algorytmu, natomiast $C$ to szukany najdłuższy wspólny podciąg (LCS) o długości $p$. Zapis $T(i:j)$ oznacza podsłowo słowa T z początkiem o indeksie $i$ oraz końcem o indeksie $j$. Przez $\bar{A}$ oznaczane będzie odwrócone słowo $A$. W opisie pracy zakładam że słowa indeksowane są od cyfry $1$.\\

Pierwszą rzeczą, którą wykonuje algorytm jest obliczenie długości najdłuższego wspólnego podciągu. Znając jego długość możemy obliczyć \textit{perfekcyjne cięcie}, które definiowany jest w następujący sposób:

\begin{definition}{}
Parę $(u, v)$ nazywamy \textit{perfekcyjnym cięciem} dla słów A, B jeżeli:
\begin{enumerate}
    \item $(u, v)$ jest prawidłowym cięciem dla słów A, B; to znaczy jeśli $LCS(A, B)$ możemy przedstawić jako $C = C_1C_2$ wtedy $C_1$ to $LCS(A(1:u), B(1:v))$ oraz $C_2$ to $LCS(A(u+1:m), B(v+1:n))$,
    \item $W(A(1:u), B(1:v)) = (m - p)/2$, gdzie $W(A, B) = |A| - |C|$
\end{enumerate}
\end{definition}

Perfekcyjne cięcie pozwala podzielić zadany problem zgodnie z definicją na dwa podproblemy: \textit{LCS(A(1:u), B(1:v))} oraz \textit{LCS(A(u+1:m), B(v+1:n))}, a z otrzymanych wyników utworzyć rozwiązanie.

Głównym problemem algorytmu jest zatem znalezienie perfekcyjnego cięcia. Do rozwiązania tego problemu wykorzystywane są dwie procedury: \textit{calmid} oraz \textit{fillone}. By lepiej zrozumieć ich ideę wprowadźmy najpierw następujące definicje:

\begin{definition}{}
$L_i(k)$ oznacza największe $h$ dla którego słowa \textit{A(i:m), B(h:n)} mają najdłuższy wspólny podciąg długości $k$.
\end{definition}

\begin{definition}
$L_i^*(k)$ oznacza najmniejsze $h$ dla którego słowa \textit{A(i:m), B(h:n)} mają najdłuższy wspólny podciąg długości $k$.
\end{definition}

\noindent W pracy przedstawiona została zależność pomiędzy \textit{perfekcyjnym cięciem (u, v)} a wartościami $L_i(k)$ oraz $L_i^*(k)$. Pozwala ona znaleźć perfekcyjne cięcie słów $A$, $B$ jeśli zna się wartości $L_i(k)$ oraz $L_i^*(k)$ dla odpowiednich $i$, $k$:

\begin{theorem}{}{perf-cut}
Para (u, v) jest \textit{perfekcyjnym cięciem} wtedy i tylko wtedy gdy $L_{u+1}(m - u - w') \geq v \geq L^*_u(u - w)$, gdzie $w = \lfloor\frac{m-p}{2}\rfloor$ oraz $w' = \lceil\frac{m-p}{2}\rceil$.
\end{theorem}

Warto wspomnieć też o zależności pomiędzy $L_i(j)$ oraz $L_i^*(j)$:
\begin{lemma}
Załóżmy że znane są wartości $L_i(j)$ dla pary odwróconych słów $(A, B)$. Wartości $L_i^*(j)$ dla nieodwróconych słów $A, B$ są wtedy zadane w następujący sposób:
\[
    L_i^*(j)_{(A, B)} = n + 1 - L_{m-i+1}(j)_{(\bar{A}, \bar{B})}
\]
\end{lemma}

Procedura \textit{calmid(A, B, m, n, x, LL, r)} oblicza wartości $L_{m-x+1-j}(j)$ za pomocą procedury \textit{fillone(A, B, m, n, R1, R2, r, s)} i zapisuje je w tablicy $LL$. Zmienna $r$ po wykonaniu funkcji oznacza natomiast największe j dla którego dana wartość jest poprawnie zdefiniowana. Procedura \textit{fillone(A, B, m, n, R1, R2, r, s)} używa dwóch tablic by obliczyć wartości $L_{s+1}(0), L_s(1), L_{s-1}(2), ...$. Obliczenia te wykonywane są iteracyjne, z wykorzystaniem poprzednich wartości. Poprawność tych funkcji oraz dokładne ich działanie opiszę przy dowodzie ich poprawności.


\section{Opis dowodu poprawności oraz złożoności algorytmu}
Prezentowany w pracy dowód poprawności algorytmu podzielony został na dowód poprawności dla każdej z procedur zaprezentowanych przez autorów. Na początek przedstawmy dowody poprawności dla procedur \textit{fillone} oraz \textit{calmid} których działanie jest kluczowe dla większości pozostałych funkcji algorytmu. Następnie przyjrzymy się metodzie znajdowania \textit{perfekcyjnego cięcia}. Pozostałe procedury działają w sposób dość jasny do zrozumienia.

\subsection{Procedura \textit{fillone}}

Tak jak zostało już wspomniane procedura \textit{fillone(A, B, m, n, $R_1, R_2$, r, s)} używa dwóch tablic by obliczyć wartości $L_{s+1}(0), L_s(1), L_{s-1}(2), ...$ . Mając zdefiniowaną pierwszą wartość $L_{s+1}(0) = n + 1$, która wynika wprost z definicji $1.2$, sposób w jaki uzyskiwane są kolejne wartości wynika z następującego lematu:
\begin{lemma}
Niech $h$ będzie największą liczbą taką że $A[i] = B[h]$ oraz $h < L_{i+1}(j - 1)$, wtedy liczba $L_i(j)$ jest definiowana przez $h$ oraz $L_{i+1}(j)$:
$$
y = \left\{ \begin{array}{ll}
h & \textrm{gdy $ L_{i+1} $ nie jest zdefiniowane}\\
L_{i+1}(j) & \textrm{gdy $ h $ nie istnieje}\\
max(h, L_{i+1}(j)) & \textrm{gdy obie wartości są zdefiniowane}
\end{array} \right.
$$
\end{lemma}

W procedurze wykorzystywana jest zmienna lokalna $lower_b$, która odpowiada za to, by nie przekroczyć wartości dla których $h$ oraz $L_{i+1}(j)$ z powyższego lematu dla danego $i, j$ są niezdefiniowane.\\
Dodatkowo wprowadzona zostaje zmienna $over$, która odpowiada za to by przerwać działanie funkcji gdy wszystkie szukane wartości zostały już obliczone.

Aby uzasadnić liniową złożoność pamięciową funkcji, wystarczy zauważyć że jedyne czego używa procedura \textit{fillone} to liniowej wielkości tablice $R_1$ oraz $R_2$, zatem złożoność pamięciowa to $O(n + m)$.
Złożoność czasowa natomiast wynika z tego że maksymalna liczba porównań którą możemy wykonać w pętli \textit{while} to $n$, ponieważ nigdy nie zwiększamy zmiennej $pos_b$, zatem złożoność to $O(n)$.

\subsection{Procedura \textit{calmid}}

Procedura składa się z dwóch części. Pierwsza to pętla for, która w $i$-tej iteracji odpowiada za wyliczanie wartości $L_{m-i+2-j}(j)$ i zapisanie jej do $R_1[j]$. Przepisywanie wyników z tablicy $R_2$ do tablicy $R_1$ na koniec każdej iteracji zapewnia, że będą znajdowały się tam poprawne wartości dla wszystkich $0 \leq j \leq r$. Poprawność tych wartości wynika z poprawności działania procedury \textit{fillone} \\
Druga część to przepisanie ostatecznych wartości $L_{m-x-j+1}(j)$ które zapisane są w tablicy $R_1(j)$ do zwracanej tablicy $LL(j)$.\\


Podobnie jak przy procedurze \textit{fillone} złożoność pamięciowa oraz czasowa są dosyć proste do uzasadnienia.
Wykorzystanie pamięci to ponownie jedynie linowego rozmiaru tablice $R_1, r_2, LL$, zatem złożoność pamięciowa to $O(n + m)$.\\
Złożoność czasowa to natomiast $(x + 1)$ wywołanie procedury \textit{fillone} w pętli for. Przypominając że złożoność procedury \textit{fillone} wynosi $O(n)$, zatem złożoność czasowa procedury \textit{calmid} to $O((x+1)n)$.

\subsection{Znajdowanie perfekcyjnego cięcia}

Procedura znajdowania \textit{perfekcyjnego cięcia} polega na dwukrotnym wywołaniu procedury calmid odpowiednio dla odwróconych słów $A, B$ oraz dla nieodwróconych. Po zakończeniu tych wywołań otrzymujemy w tablicach $LL_1, LL_2$ wyliczone wartości odpowiednio: $n + 1 - L_{m-w+1-k}(k)$ dla $k \leq r_1$ oraz $L_{w+k+1}(m-(w+k)-w')$\ $(= L_{m-w'+1-(p-k)}(p-k))$ dla $p-k \leq r_2$. Korzystając z lematu $1.1$ możemy otrzymać z wartości $n + 1 - L_{m-w+1-k}(k)$ odpowiednie wartości by móc przy pomocy twierdzenia 1.1 stwierdzić, że wystarczy dobrać $u = k + w$ oraz $v = LL_1(k)$ by dostać poprawne perfekcyjne cięcie.\\
Złożoność czasowa i pamięciowa to ta sama, która miała procedura calmid, gdyż dodatkowo wykonywana jest jedynie pętla \textit{for} by dobrać odpowiednie $(u, v)$, która ma złożoność $O(r_1) \leq O(n)$.

\subsection{Złożoność czasowa i pamięciowa algorytmu}
Patrząc na główną procedurę \textit{Longest common subsequence}, która działa zgodnie z techniką \textit{dziel i zwyciężaj} opisaną na początku dokumentu możemy obliczyć złożoność czasową algorytmu.
Mamy do rozpatrzenia dwa przypadki:
\begin{itemize}
    \item Przypadku bazowego: jego rozwiązanie to wywołanie funkcji \textit{calmid} z parametrem $x = m-p$, gdzie $m - p \leq 2$, zatem jest to złożoność $O(n)$ oraz dodatkowe operacje w pętlach \textit{while}, których maksymalnie zostanie wykonanych $O(m)$. Zatem całkowita złożoność przypadku bazowego to $O(n + m)$.
    \item Znajdowanie \textit{perfekcyjnego cięcia} oraz dwa wywołania rekurencyjne: pierwsze to dwa wywołania procedury \textit{calmid} z parametrem $x = w$ lub $x = w'$ odpowiednio, zatem ich złożoność to $O(n(w + 1))$ i $O(n(w' + 1))$. Złożoność ta to inaczej $O(n(m - p))$, ponieważ tak dobrany był parametry $w$ i $w'$. Dodając dwa wywołania rekurencyjne zgodne z \textit{perfekcyjnym cięciem} $(u, v)$ dostajemy złożoność:
    \[
        T(m, n, p) = O(n(m - p)) + T(u, v, u - w) + T(m - u, n - v, m - u - w')
    \]
    która po przeliczeniach okazuje się wynosić $O(n(m - p))$, ponieważ musi tak być w przypadku gdy $m-p \leq 2$, ponieważ $p \leq n$.
\end{itemize}

Jedyną operacją wykonywaną poza procedurą \textit{Longest common subsequence} jest znalezienie na początku algorytmu długości najdłuższego wspólnego podciągu, która również wykonywana jest w czasie $O(n(m - p))$.
Podsumowując całkowita złożoność czasowa algorytmu to:
\[
O(n(m - p)) + O(n + m) + O(n(m - p)) = O(n(m - p))
\]\\

Złożoność pamięciowa algorytmu jest natomiast liniowa względem rozmiaru słów $A$ i $B$, czyli wynosi $O(n + m)$. By uzyskać taką złożoność przy wywołaniach rekurencyjnych zamiast podsłów przekazujemy jedynie indeksy początku i końca podsłowa. Tablice $LL_1$ oraz $LL_2$ wykorzystywane do obliczania i przechowywanych wartości mają zawsze wielkość liniową. Maksymalna głębokość rekursji jaka może wystąpić w algorytmie to $O(log(m-p))$, zatem całkowita złożoność pamięciowa algorytmu pozostaje liniowa.

\end{document}
